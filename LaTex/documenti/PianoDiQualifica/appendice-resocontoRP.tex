\subsection{Revisione di Progettazione}
Sono stati verificati i documenti prodotti mediante le procedure descritte nelle \NormeDiProgetto{}, nella sezione relativa alle procedure di supporto dei processi di \VV{}.\\
Si è proceduto nel modo seguente:\begin{itemize}
	\item gli errori rilevati con più frequenza sono stati integrati alla lista di controllo, allegata in appendice al documento \NormeDiProgetto{};
	\item sono stati corretti gli errori;
	\item è stato applicato il ciclo PDCA per il miglioramento del processo di verifica;
	\item è stata applicata l'\glossario{inspection} utilizzando la lista di controllo;
	\item sono state calcolate le metriche per i documenti descritte in \NormeDiProgetto{};
	\item è stato verificato il tracciamento delle componenti rispetto ai requisiti;
	\item è stato verificato il tracciamento dei test rispetto alle componenti e ai requisiti.
\end{itemize}
L'avanzamento dei processi è stato verificato sulla base di quanto riportato nelle \NormeDiProgetto{}. Sono quindi state calcolate le metriche di processo e confrontate con gli obiettivi descritti in \sezione{sec:metriche_processo}.

\subsubsection{Esiti delle verifiche}
\paragraph{Indice di Gulpease}\mbox{}\\
\begin{longtable}{|c|c|c|c|c|}
	\hline \multicolumn{1}{|c|}{\textbf{Documento}} & \multicolumn{1}{c|}{\textbf{Risultato}} & \multicolumn{1}{c|}{\textbf{Accettazione}} & \multicolumn{1}{c|}{\textbf{Ottimalità}} & \multicolumn{1}{c|}{\textbf{Esito}}\\
	\hline 
	\endfirsthead
	
	\hline \multicolumn{1}{|c|}{\textbf{Documento}} & \multicolumn{1}{c|}{\textbf{Risultato}} & \multicolumn{1}{c|}{\textbf{Accettazione}} & \multicolumn{1}{c|}{\textbf{Ottimalità}} & \multicolumn{1}{c|}{\textbf{Esito}}\\
	\hline 
	\endhead
	
	\hline \multicolumn{5}{|r|}{\ToBeContinued} \\ 
	\hline
	\endfoot
	
	\endlastfoot
	
	\hline \NormeDiProgetto{} & 77 & 40-100 & 50-100 & Superato\\
	\hline \PianoDiProgetto{} & 72 & 40-100 & 50-100 & Superato \\
	\hline \PianoDiQualifica{} & 76 & 40-100 & 50-100 & Superato \\
	\hline \AnalisiDeiRequisiti{} & 90 & 40-100 & 50-100 & Superato \\
	\hline \Glossario{} & 65 & 40-100 & 50-100 & Superato \\
	\hline \SpecificaTecnica{} & 72 & 40-100 & 50-100 & Superato\\
	\hline VerbaleEsterno8\_12\_2016 & 79 & 40-100 & 50-100 & Superato \\
	\hline VerbaleInterno9\_12\_2016 & 80 & 40-100 & 50-100 & Superato \\
	\hline VerbaleInterno16\_12\_2016 & 83 & 40-100 & 50-100 & Superato \\
	\hline VerbaleInterno22\_12\_2016 & 85 & 40-100 & 50-100 & Superato \\
	\hline VerbaleEsterno23\_12\_2016 & 75 & 40-100 & 50-100 & Superato \\
	\hline VerbaleInterno30\_12\_2016 & 84 & 40-100 & 50-100 & Superato \\
	\hline VerbaleInterno30\_01\_2016 & 85 & 40-100 & 50-100 & Superato \\
	\hline VerbaleInterno23\_02\_2016 & 83 & 40-100 & 50-100 & Superato \\
	\hline VerbaleInterno2\_03\_2016 & 73 & 40-100 & 50-100 & Superato \\
	\hline
	\caption{Valori indice di Gulpease - Revisione di Progettazione}
\end{longtable}
I valori ottenuti sono migliorati rispetto alla scorsa revisione, in quanto è stato riscontrato e corretto un errore nel calcolo dei caratteri nello script utilizzato. Emerge quindi come l'\AnalisiDeiRequisiti{} abbia un valore molto alto, dettato dalla sua natura ad elenco e tabellare.

\paragraph{Fallimento dei test}\mbox{}\\
La percentuale di test falliti sulla compilazione dei documenti ottenuta è del 22\%, un valore non accettabile secondo le metriche imposte da questo documento.\\
Questo dato denota una poca attenzione nel verificare che le proprie modifiche non producano errori di compilazione in \LaTeX{}.

\paragraph{Metriche di progettazione}\mbox{}\\
Vengono ora presentati i risultati ottenuti dalle misurazioni relativamente alle metriche di progettazione:
\begin{longtable}{|c|c|c|c|c|}
	\hline \multicolumn{1}{|c|}{\textbf{Metrica}} & \multicolumn{1}{c|}{\textbf{Risultato}} & \multicolumn{1}{c|}{\textbf{Accettazione}} & \multicolumn{1}{c|}{\textbf{Ottimalità}} & \multicolumn{1}{c|}{\textbf{Esito}}\\
	\hline 
	\endfirsthead
	
	\hline \multicolumn{1}{|c|}{\textbf{Documento}} & \multicolumn{1}{c|}{\textbf{Risultato}} & \multicolumn{1}{c|}{\textbf{Accettazione}} & \multicolumn{1}{c|}{\textbf{Ottimalità}} & \multicolumn{1}{c|}{\textbf{Esito}}\\
	\hline 
	\endhead
	
	\hline \multicolumn{5}{|r|}{\ToBeContinued} \\ 
	\hline
	\endfoot
	
	\endlastfoot
	
	\hline Fan-in max (OrderGateway) & 5 & >=0 & >=2 & Superato \\
	\hline Fan-in medio & 0.65 & >=0 & >=2 & Superato \\
	\hline Fan-out max (GUI) & 9 & 0-5 & 0-1 & Non superato \\
	\hline Fan-out medio & 0.63 & 0-5 & 0-1 & Superato\\
	\hline
	\caption{Esiti metriche di progettazione architetturale - Revisione di Progettazione}
\end{longtable}