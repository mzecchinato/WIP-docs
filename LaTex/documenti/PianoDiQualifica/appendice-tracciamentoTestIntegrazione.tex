\section{Tracciamento Test}

\subsection{Test di Integrazione}

\subsubsection{Framework}

\begin{longtable}{|c|P{9cm}|c|}
	\hline \multicolumn{1}{|l|}{\textbf{Codice}} &  \multicolumn{1}{l|}{\textbf{Descrizione}} & \multicolumn{1}{l|}{\textbf{Componente}} \\ 
	\endfirsthead
	\hline \test{I} & Verificare che la bubble riceva gli input dalla GUI & Controller::BubbleGenerica \\
	\hline \test{I} & Verificare che la GUI riceva i dati dalla bubble &  \\
	\hline \test{I} & Verificare che la GUI venga modificata come previsto & View::Gui \\
	\hline \test{I} & Verificare che l'elemento funzionale riceva i dati & Bubble memory \\
	\hline \test{I} & Verificare la bubble riceva il risultato dell'elaborazione &  \\
	\hline \test{I} & Verificare che il risultato ritornato alla bolla sia corretto &  \\
	\hline \test{I} & Verificare che inviando all'elemento funzionale dati non corretti sia notificato l'errore & Bubble memory \\
	\hline \test{I} & Verificare che i dati inseriti correttamente modifichino la GUI nel modo atteso & View::Gui \\
	\hline \test{I} & Verificare che i dati inseriti non corretti non modifichino la GUI e sia notificato un errore &  \\
	\hline \test{I} & Verificare che un evento originato dall'elemento funzionale abbia ripercussioni sulla bubble generica e sulla GUI &  \\
	\hline \test{I} & Verificare che gli elementi funzionali ricevano dati dalla bubble & Bubble memory \\
	\hline \test{I} & Verificare che la bubble riceva il risultato dell'elaborazione dei dati &  \\
	\hline \test{I} & Verificare che all'arrivo di un segnale dalla GUI sia ritornato un valore corretto alla bubble & Controller::BubbleGenerica \\
	\hline \test{I} & Verificare che con l'invio di dati non corretti all'elemento funzionale sia notificato l'errore & Bubble memory \\
	\hline \test{I} & Verificare che la bubble riceva input dalla GUI & Controller::BubbleGenerica \\
	\hline \test{I} & Verificare che la GUI riceva dati dalla bubble & \\
	\hline \test{I} & Verificare che con la modifica dell'elemento funzionale la GUI venga modificata & View::Gui \\
	\hline
	\caption{Test di integrazione per il framework}
\end{longtable}

\subsubsection{Bubble To-do List}

\begin{longtable}{|c|P{9cm}|c|}
	\hline \multicolumn{1}{|l|}{\textbf{Codice}} &  \multicolumn{1}{l|}{\textbf{Descrizione}} & \multicolumn{1}{l|}{\textbf{Componente}} \\ 
	\endfirsthead
	\hline \test{I} & Verificare che con la creazione di un nuovo elemento la vista venga modificata &  \\
	\hline \test{I} & Verificare che con la creazione di un nuovo elemento con dati non corretti il controller ritorni un messaggio d'errore &  \\
	\hline \test{I} & Verificare che con la creazione di un nuovo elemento con dati corretti ma con fallimento del salvataggio nella bubble memory venga notificato un messaggio d'errore &  \\
	\hline \test{I} & Verificare che alla creazione di una notifica statica con dati corretti sia ritornato un messaggio dell'avvenuta creazione &  \\
	\hline \test{I} & Verificare che alla creazione di una notifica statica con dati non corretti sia ritornato un messaggio d'errore  &  \\
	\hline \test{I} & Verificare che alla creazione di un nuovo elemento nella view esso venga correttamente inviato al controller &  \\
	\hline \test{I} & Verificare che alla cancellazione di un elemento nella view questo venga correttamente inviato al controller &  \\
	\hline \test{I} & Verificare alla creazione di una notifica statica nella view essa venga correttamente inviata al controller &  \\
	\hline \test{I} & Verificare che un errore proveniente dal controller venga renderizzato in modo corretto dalla view &  \\
	\hline \test{I} & Verificare che l'aggiornamento dell'interfaccia grafica venga interpretato correttamente e che l'interfaccia grafica subisca le variazioni attese &  \\
	\hline
	\caption{Test di integrazione per la bubble To-do list}
\end{longtable}

\subsubsection{Bubble \& eat}

\begin{longtable}{|c|P{9cm}|c|}
	\hline \multicolumn{1}{|l|}{\textbf{Codice}} &  \multicolumn{1}{l|}{\textbf{Descrizione}} & \multicolumn{1}{l|}{\textbf{Componente}} \\ 
	\endfirsthead
	\hline \test{I} & Verificare che alla visualizzazione del menu da parte del cliente l'Order Gateway ritorni i dati correttamente & Demo Bubble\&eat::Ristorante::Order Gateway::Order Gateway \\
	\hline \test{I} & Verificare che alla visualizzazione del menu da parte del cliente senza la presenza di un menu nell'Order Gateway venga ritornato un messaggio d'errore all'utente & Demo Bubble\&eat::Ristorante::Order Gateway::Order Gateway \\
	\hline \test{I} & Verificare che l'invio di un ordinazione da parte di un utente sia immagazzinata all'interno del database e che venga correttamente inoltrata al cuoco & Demo Bubble\&eat::Ristorante::Order Gateway::Order Gateway \\
	\hline \test{I} & Verificare che con l'invio di un ordinazione da parte di un cliente con dati sbagliati l'ordinazione non venga salvata, non venga inoltrata il cuoco e venga invece ritornato all'utente un messaggio di errore & Demo Bubble\&eat::customer::Bubble customer \\
	\hline \test{I} & Verificare che con l'invio dei dati personali con dati corretti i dati vengano correttamente salvati nel database & Demo Bubble\&eat::Ristorante::Order Gateway::Order Gateway  \\
	\hline \test{I} & Verificare che con l'invio di dati personali con dati non corretti i dati non vengono salvati nel database e venga inviata una segnalazione di errore all'utente & Demo Bubble\&eat::customer::Bubble customer  \\
	
	\hline \test{I} & Verificare che con l'arrivo di un'ordinazione al Cuoco l'ordinazione sia inserita nella to-do list del Cuoco &  Demo Bubble\&eat::Ristorante::chef::Bubble chef \\
	\hline \test{I} & Verificare che con il completamento della preparazione di un piatto da parte del Cuoco all'interno dell'ordine impostato il piatto come pronto & Demo Bubble\&eat::Ristorante::Order Gateway::Ordinazione::Order \\
	\hline \test{I} & Verificare che quando lo stato di un piatto diventa \virgolette{pronto} l'ordinazione venga trasferita al Fattorino & Demo Bubble\&eat::Ristorante::deliveryman::Bubble deliveryman \\
	\hline \test{I} & Verificare che con il completamento della preparazione di un piatto da parte del Cuoco con errore nel salvataggio venga mostrato al Cuoco un messaggio d'errore & Demo Bubble\&eat::Ristorante::chef::Bubble chef \\
	\hline \test{I} & Verificare che al segnale di piatto pronto la consegna sia visualizzata nella lista del Fattorino & Demo Bubble\&eat::Ristorante::deliveryman::Bubble deliveryman  \\
	\hline \test{I} & Verificare che al segnale di completamento della consegna da parte del Fattorino il segnale arrivi correttamente correttamente all'Order Gateway &  Demo Bubble\&eat::Ristorante::Order Gateway::Order Gateway \\
	\hline \test{I} & Verificare che al segnale di non completamento della consegna da parte del Fattorino il segnale arrivi correttamente all’Order Gateway &  Demo Bubble\&eat::Ristorante::Order Gateway::Order Gateway \\
	\hline \test{I} & Verificare che al segnale di carenza di un ingrediente l'ingrediente mancante sia correttamente visualizzato nella lista del Responsabile Acquisti & DemoBubble\&eat::Ristorante::PurchasingManager::BubblePurchasingManager  \\
	\hline \test{I} & Verificare che al segnale di ingrediente acquistato da parte del Responsabile Acquisti il dato sia correttamente salvato nel database e vengano incrementati i valori relativi a tale ingrediente nella lista delle scorte & Demo Bubble\&eat::Ristorante::Order Gateway::Order Gateway  \\
	\hline \test{I} & Verificare che al segnale di carenza di un ingrediente da parte del Direttore all'Order Gateway vengano correttamente salvate le informazioni nel database & Demo Bubble\&eat::Ristorante::Order Gateway::Order Gateway  \\
	\hline \test{I} & Verificare che al segnale di cancellazione di un ordine da parte del Direttore le informazioni siano correttamente salvate nel database & Demo Bubble\&eat::Ristorante::Order Gateway::Order Gateway  \\
	\hline \test{I} & Verificare che al segnale di creazione di un nuovo menu nel database sia salvato il menu correttamente nel database & Demo Bubble\&eat::Ristorante::Order Gateway::Order Gateway  \\
	\hline \test{I} & Verificare che alla modifica del menu da parte del Direttore sia aggiornato correttamente il menu nel database & Demo Bubble\&eat::Ristorante::Order Gateway::Order Gateway  \\
	\hline
	\caption{Test di integrazione per la Bubble \& eat}
\end{longtable}
