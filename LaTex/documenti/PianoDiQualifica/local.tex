%%%%%%%%%%%%%%
%  COSTANTI  %
%%%%%%%%%%%%%%
\usepackage{amsmath}

% In questa prima parte vanno definite le 'costanti' utilizzate soltanto da questo documento.
% Devono iniziare con una lettera maiuscola per distinguersi dalle funzioni.

\newcommand{\DocTitle}{Piano di Qualifica}
\newcommand{\DocVersion}{\VersionePQ{}}

\newcommand{\DocRedazione}{\alice \\ \bea \\ \marco \\ \mattia}
\newcommand{\DocVerifica}{\lorenzo \\ \nick}
\newcommand{\DocApprovazione}{\mattia}

\newcommand{\DocUso}{Esterno}
\newcommand{\DocDistribuzione}{
	\Committente{} \\
	Gruppo \GroupName{} \\
	\Proponente{}
}

% La descrizione del documento
\newcommand{\DocDescription}{
 Documento riguardante le strategie di verifica e gli obiettivi qualitativi per il progetto \ProjectName{} adottate dal gruppo \GroupName.
 }

%%%%%%%%%%%%%%
%  FUNZIONI  %
%%%%%%%%%%%%%%

% In questa seconda parte vanno definite le 'funzioni' utilizzate soltanto da questo documento.
\usepackage{xr}
\externaldocument[AdR-]{../AnalisiDeiRequisiti/capitolo-requisiti}

\makeatletter
\newcommand*{\textlabel}[2]{%
	\edef\@currentlabel{#1}% Set target label
	\phantomsection% Correct hyper reference link
	\label{#2}% store label
}
\makeatother

\newcounter{testcounter}
\newcommand*{\test}[2]{%
	\refstepcounter{testcounter}
	\textlabel{T#1\thetestcounter}{#2}
	T#1\thetestcounter
}

\newcounter{testcount}
\newcommand*{\testt}{%
	\refstepcounter{testcount}
	TI\thetestcount
}


\definecolor{green}{RGB}{0,180,0}
\definecolor{red}{RGB}{215,0,0}
\newcommand{\si}{\textcolor{green}{Superato}\stepcounter{si}}
\newcommand{\no}{\textcolor{red}{Non implementato}\stepcounter{no}}

\newcounter{no}
\newcounter{si}