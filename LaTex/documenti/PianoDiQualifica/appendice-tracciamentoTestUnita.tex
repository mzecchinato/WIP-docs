\subsection{Test di Unità}

%\subsubsection{Framework}
%
%\begin{longtable}{|P{5cm}|P{5cm}|P{5cm}|}
%	\hline \textbf{Codice} & \textbf{Descrizione} & \textbf{Requisito} \\
%	\endfirshead
%	
%	\hline \test{S} & Data una GUI con più elementi e un elemento funzionale, simulare un input utente e far inviare un segnale dalla GUI alla bubble, seguendo tutto il percorso, per controllare che ciò che la GUI ritorna sia corretto & \\
%	\hline \test{S} & Data una GUI con più elementi e un elemento funzionale, simulare un input utente e far inviare un segnale non corretto dalla GUI alla bubble, seguendo tutto il percorso, per controllare che la GUI ritorni un errore e non si aggiorni le informazioni sbagliate & \\
%	\hline \test{S} & Data una GUI con più elementi e un elemento funzionale, simulare un evento originato dall'elemento funzionale e verificare che questo abbia ripercussioni sulla bubble generica e sulla GUI & \\
%	\hline \test{S} & Data una GUI con un elemento e più elementi funzionali, simulare un input utente e far inviare un segnale dalla GUI alla bubble, tracciando tutto il percorso, per controllare che ciò che la GUI ritorna sia corretto & \\
%	\hline \test{S} & Data una GUI con un elemento e più elementi funzionali, simulare un input utente e far inviare un segnale non corretto dalla GUI alla bubble, seguendo tutto il percorso, per controllare che la GUI ritorni un errore e non si aggiorni le informazioni sbagliate & \\
%	\hline \test{S} & Data una GUI con un elemento e più elementi funzionali, simulare un evento originato dall'elemento funzionale e verificare che questo abbia ripercussioni sulla bubble generica e sulla GUI & \\
%	\hline \test{S} & Data una GUI con più elementi e più elementi funzionali, simulare un input utente e far inviare un segnale dalla GUI alla bubble, tracciando tutto il percorso, per controllare che ciò che la GUI ritorna sia corretto & \\
%	\hline \test{S} & Data una GUI con più elementi e più elementi funzionali, simulare un input utente e far inviare un segnale non corretto dalla GUI alla bubble, seguendo tutto il percorso, per controllare che la GUI ritorni un errore e non si aggiorni le informazioni sbagliate & \\
%	\hline \test{S} & Data una GUI con un elemento e più elementi funzionali, simulare un evento originato dall'elemento funzionale e verificare che questo abbia ripercussioni sulla bubble generica e sulla GUI & \\
%	\hline
%	\caption{Test di sistema per il framework}
%\end{longtable}

\subsubsection{Order Gateway}

\begin{longtable}{|c|P{9cm}|c|}
	\hline \multicolumn{1}{|l|}{\textbf{Codice}} &  \multicolumn{1}{l|}{\textbf{Descrizione}} & \multicolumn{1}{l|}{\textbf{Esito}} \\ 
	\endfirsthead
	\hline \test{U} & Verifica che il server:
	\begin{itemize}
		\item viene creato correttamente;
		\item ascolta alla porta assegnata;
		\item ritorna un socket che non sia nullo.
	\end{itemize}
	& Superato \\
	\hline\test{U} & Verifica che l'handler dell'admin:
	\begin{itemize}
		\item si connette correttamente alla porta;
		\item ottiene il menu;
		\item aggiuge un piatto al menu;
		\item modifica un piatto dal menu;
		\item rimuove un piatto dal menu;
		\item ottiene tutte le ordinazioni;
		\item ottiene le ordinazioni attive;
		\item ottiene le ordinazioni completate;
		\item rimuove un'ordinazione.
	\end{itemize}
	 & Superato \\
	 \hline\test{U} & Verifica che l'handler del client:
	 \begin{itemize}
	 	\item si connette correttamente alla porta;
	 	\item ottiene il menu;
	 	\item crea una nuova ordinazione;
	 	\item ottiene i dati relativo al proprio ordine;
	 	\item ottiene l'id relativo al proprio ordine;
	 	\item riceve una notifica quando l'ordine è completato;
	 	\item riceva un errore quando non viene trovato l'ordine relativo al prorprio id;
	 	\item riottiene lo stato del prorpio ordine nel caso di una disconnessione;
	 	\item ottiene la notifica del prorpio ordine se viene completato mentre l'utente è disconnesso alla sua riconnessione.
	 \end{itemize}
	 & Superato \\
	 \hline\test{U} & Verifica che l'handler dell'cook:
	 \begin{itemize}
	 	\item si connette correttamente alla porta;
	 	\item si disconnette correttamente;
	 	\item comunica correttamente con il database;
	 	\item controlla che riceva correttamente le ordinazioni attive(anche se non ce ne fossero) quando è nello stato di pronto;
	 	\item attiva correttamente il completamento dell'ordinazione.
	 \end{itemize}
 	& Superato \\
	\hline\test{U} & Verifica che l'oggetto JSON dedicato alla connessione sia formattato in modo corretta & Superato \\
	\hline \test{U} & Verifica che il CookReducer:
	\begin{itemize}
		\item legge che lo stato è indefinito se non ancora istanziato;
		\item imposta in modo corretto lo stato in assente;
		\item imposta in modo corretto lo stato in presente.
	\end{itemize}
	& Superato \\
	\hline \test{U} & Verifica che il MenuReducer:
	\begin{itemize}
		\item legge che lo stato è indefinito se non ancora istanziato;
		\item aggiunge correttamente il piatto ed incrementa l'id;
		\item rimuove in modo corretto un piatto dal menù;
		\item modifica correttamente i dati di un piatto senza modificarne l'id.
	\end{itemize}
	& Superato \\
	\hline \test{U} & Verifica che il OrderReducer:
	\begin{itemize}
		\item aggiunge correttamente un'ordinazione;
		\item cambia in modo corretto lo stato di un'ordinazione completata;
		\item rimuove in modo corretto un'ordinazione.
	\end{itemize}
	& Superato \\
	\hline\test{U} & Verifica che l'istanza di un oggetto di tipo Actions sia formattato in modo corretto & Superato \\
	\hline\test{U} & Verifica che le istanze di tipo Actions della presenza e dell'assenza del cook sono generate in modo corretto & Superato \\
	\hline\test{U} & Verifica che le istanze di tipo Actions di aggiunta,modifica e rimozione del menu sono generate in modo corretto & Superato \\
	\hline\test{U} & Verifica che le istanze di tipo Actions di aggiunta,modifica e rimozione dell'ordinazione sono generate in modo corretto & Superato \\
	\hline
	\caption{Test di unità per l'order gateway}
\end{longtable}

