\subsection{Test di Unità}

%\subsubsection{Framework}                                            %
%\begin{longtable}{|P{5cm}|P{5cm}|P{5cm}|}
%	\hline \textbf{Codice} & \textbf{Descrizione} & \textbf{Requisito} \\
%	\endfirshead
%	
%	\hline \test{S} & Data una GUI con più elementi e un elemento funzionale, simulare un input utente e far inviare un segnale dalla GUI alla bubble, seguendo tutto il percorso, per controllare che ciò che la GUI ritorna sia corretto & \\
%	\hline \test{S} & Data una GUI con più elementi e un elemento funzionale, simulare un input utente e far inviare un segnale non corretto dalla GUI alla bubble, seguendo tutto il percorso, per controllare che la GUI ritorni un errore e non si aggiorni le informazioni sbagliate & \\
%	\hline \test{S} & Data una GUI con più elementi e un elemento funzionale, simulare un evento originato dall'elemento funzionale e verificare che questo abbia ripercussioni sulla bubble generica e sulla GUI & \\
%	\hline \test{S} & Data una GUI con un elemento e più elementi funzionali, simulare un input utente e far inviare un segnale dalla GUI alla bubble, tracciando tutto il percorso, per controllare che ciò che la GUI ritorna sia corretto & \\
%	\hline \test{S} & Data una GUI con un elemento e più elementi funzionali, simulare un input utente e far inviare un segnale non corretto dalla GUI alla bubble, seguendo tutto il percorso, per controllare che la GUI ritorni un errore e non si aggiorni le informazioni sbagliate & \\
%	\hline \test{S} & Data una GUI con un elemento e più elementi funzionali, simulare un evento originato dall'elemento funzionale e verificare che questo abbia ripercussioni sulla bubble generica e sulla GUI & \\
%	\hline \test{S} & Data una GUI con più elementi e più elementi funzionali, simulare un input utente e far inviare un segnale dalla GUI alla bubble, tracciando tutto il percorso, per controllare che ciò che la GUI ritorna sia corretto & \\
%	\hline \test{S} & Data una GUI con più elementi e più elementi funzionali, simulare un input utente e far inviare un segnale non corretto dalla GUI alla bubble, seguendo tutto il percorso, per controllare che la GUI ritorni un errore e non si aggiorni le informazioni sbagliate & \\
%	\hline \test{S} & Data una GUI con un elemento e più elementi funzionali, simulare un evento originato dall'elemento funzionale e verificare che questo abbia ripercussioni sulla bubble generica e sulla GUI & \\
%	\hline
%	\caption{Test di sistema per il framework}
%\end{longtable}

\subsubsection{Framework}

\begin{longtable}{|c|P{9cm}|c|}
	\hline \multicolumn{1}{|l|}{\textbf{Codice}} &  \multicolumn{1}{l|}{\textbf{Descrizione}} & \multicolumn{1}{l|}{\textbf{Esito}} \\ 
	\endfirsthead
	\hline \test{U} & Verifica che il DataBase:
	\begin{itemize}
		\item abbia l'URL corretto;
		\item si connetta e riceva l'oggetto in modo corretto;
		\item salvi gli oggetti nella collezzione nel modo corretto.
	\end{itemize}
	& Superato \\
	\hline \test{U} & Verifica che l'ExternalAPI:
	\begin{itemize}
		\item controlli che esita un myurl valido;
		\item riceva un un oggetto JSON valido dall'url.
	\end{itemize}
	& Superato \\
	\hline \test{U} & Verifica che l'ItemStore:
	\begin{itemize}
		\item venga creato in modo coretto e sia vuoto;
		\item aggiunga un elemento in modo corretto;
		\item rimuova correttamente l'ultimo elemento.
	\end{itemize}
	& Superato \\
	\hline \test{U} & Verifica che il LifeCycle:
	\begin{itemize}
		\item venga creato in modo coretto;
		\item faccia il reset del timer correttamente.
	\end{itemize}
	& Superato \\
	\hline \test{U} & Verifica che la classe MatchRegularExpr:
	\begin{itemize}
		\item venga instanziata in modo coretto;
		\item utilizzi le funzioni di match in modo corretto;
		\item il risultato del match sia corretto.
	\end{itemize}
	& Superato \\
	\hline \test{U} & Verifica che l'InputFile sia sia formato in modo corretto & Superato \\
	& Superato \\
	\hline \test{U} & Verifica che l'InputText sia sia formato in modo corretto & Superato \\
	\hline \test{U} & Verifica che il Label:
	\begin{itemize}
		\item venga renderizzato;
		\item riceva il valore da renderizzare correttamente.
	\end{itemize}
	& Superato \\
	\hline \test{U} & Verifica che il TextView:
	\begin{itemize}
		\item venga renderizzato;
		\item riceva il valore da renderizzare correttamente.
	\end{itemize}
	& Superato \\
	\hline
	\caption{Test di unità per Bubble \& eat}
\end{longtable}

\subsubsection{Order Gateway}

\begin{longtable}{|c|P{9cm}|c|}
	\hline \multicolumn{1}{|l|}{\textbf{Codice}} &  \multicolumn{1}{l|}{\textbf{Descrizione}} & \multicolumn{1}{l|}{\textbf{Esito}} \\ 
	\endfirsthead
	\hline \test{U} & Verifica che il server:
	\begin{itemize}
		\item venga creato correttamente;
		\item ascolti alla porta assegnata;
		\item ritorni un socket che non sia nullo.
	\end{itemize}
	& Superato \\
	\hline\test{U} & Verifica che l'handler dell'admin:
	\begin{itemize}
		\item si connetta correttamente alla porta;
		\item ottenga il menu;
		\item aggiuga un piatto al menu;
		\item modifichi un piatto dal menu;
		\item rimuova un piatto dal menu;
		\item ottenga tutte le ordinazioni;
		\item ottenga le ordinazioni attive;
		\item ottenga le ordinazioni completate;
		\item rimuova un'ordinazione.
	\end{itemize}
	 & Superato \\
	 \hline\test{U} & Verifica che l'handler del client:
	 \begin{itemize}
	 	\item si connette correttamente alla porta;
	 	\item ottenga il menu;
	 	\item crei una nuova ordinazione;
	 	\item ottenga i dati relativo al proprio ordine;
	 	\item ottenga l'id relativo al proprio ordine;
	 	\item riceva una notifica quando l'ordine è completato;
	 	\item riceva un errore quando non viene trovato l'ordine relativo al prorprio id;
	 	\item riottenga lo stato del prorpio ordine nel caso di una disconnessione;
	 	\item ottenga la notifica del prorpio ordine se viene completato mentre l'utente è disconnesso alla sua riconnessione.
	 \end{itemize}
	 & Superato \\
	 \hline\test{U} & Verifica che l'handler dell'cook:
	 \begin{itemize}
	 	\item si connetta correttamente alla porta;
	 	\item si disconnetta correttamente;
	 	\item comunichi correttamente con il database;
	 	\item controlli che riceva correttamente le ordinazioni attive(anche se non ce ne fossero) quando è nello stato di pronto;
	 	\item attivi correttamente il completamento dell'ordinazione.
	 \end{itemize}
 	& Superato \\
	\hline\test{U} & Verifica che l'oggetto JSON dedicato alla connessione sia formattato in modo corretta & Superato \\
	\hline \test{U} & Verifica che il CookReducer:
	\begin{itemize}
		\item legga che lo stato è indefinito se non ancora istanziato;
		\item imposti in modo corretto lo stato in assente;
		\item imposti in modo corretto lo stato in presente.
	\end{itemize}
	& Superato \\
	\hline \test{U} & Verifica che il MenuReducer:
	\begin{itemize}
		\item legga che lo stato è indefinito se non ancora istanziato;
		\item aggiuga correttamente il piatto ed incrementa l'id;
		\item rimuova in modo corretto un piatto dal menù;
		\item modifichi correttamente i dati di un piatto senza modificarne l'id.
	\end{itemize}
	& Superato \\
	\hline \test{U} & Verifica che l'OrderReducer:
	\begin{itemize}
		\item aggiuga correttamente un'ordinazione;
		\item cambi in modo corretto lo stato di un'ordinazione completata;
		\item rimuova in modo corretto un'ordinazione.
	\end{itemize}
	& Superato \\
	\hline\test{U} & Verifica che l'istanza di un oggetto di tipo Actions sia formattato in modo corretto & Superato \\
	\hline\test{U} & Verifica che le istanze di tipo Actions della presenza e dell'assenza del cook sono generate in modo corretto & Superato \\
	\hline\test{U} & Verifica che le istanze di tipo Actions di aggiunta,modifica e rimozione del menu sono generate in modo corretto & Superato \\
	\hline\test{U} & Verifica che le istanze di tipo Actions di aggiunta,modifica e rimozione dell'ordinazione sono generate in modo corretto & Superato \\
	\hline
	\caption{Test di unità per l'order gateway}
\end{longtable}

