\subsection{Test di sistema}

%\subsubsection{Framework}
%
%\begin{longtable}{|P{5cm}|P{5cm}|P{5cm}|}
%	\hline \textbf{Codice} & \textbf{Descrizione} & \textbf{Requisito} \\
%	\endfirshead
%	
%	\hline \test{S} & Data una GUI con più elementi e un elemento funzionale, simulare un input utente e far inviare un segnale dalla GUI alla bubble, seguendo tutto il percorso, per controllare che ciò che la GUI ritorna sia corretto & \\
%	\hline \test{S} & Data una GUI con più elementi e un elemento funzionale, simulare un input utente e far inviare un segnale non corretto dalla GUI alla bubble, seguendo tutto il percorso, per controllare che la GUI ritorni un errore e non si aggiorni le informazioni sbagliate & \\
%	\hline \test{S} & Data una GUI con più elementi e un elemento funzionale, simulare un evento originato dall'elemento funzionale e verificare che questo abbia ripercussioni sulla bubble generica e sulla GUI & \\
%	\hline \test{S} & Data una GUI con un elemento e più elementi funzionali, simulare un input utente e far inviare un segnale dalla GUI alla bubble, tracciando tutto il percorso, per controllare che ciò che la GUI ritorna sia corretto & \\
%	\hline \test{S} & Data una GUI con un elemento e più elementi funzionali, simulare un input utente e far inviare un segnale non corretto dalla GUI alla bubble, seguendo tutto il percorso, per controllare che la GUI ritorni un errore e non si aggiorni le informazioni sbagliate & \\
%	\hline \test{S} & Data una GUI con un elemento e più elementi funzionali, simulare un evento originato dall'elemento funzionale e verificare che questo abbia ripercussioni sulla bubble generica e sulla GUI & \\
%	\hline \test{S} & Data una GUI con più elementi e più elementi funzionali, simulare un input utente e far inviare un segnale dalla GUI alla bubble, tracciando tutto il percorso, per controllare che ciò che la GUI ritorna sia corretto & \\
%	\hline \test{S} & Data una GUI con più elementi e più elementi funzionali, simulare un input utente e far inviare un segnale non corretto dalla GUI alla bubble, seguendo tutto il percorso, per controllare che la GUI ritorni un errore e non si aggiorni le informazioni sbagliate & \\
%	\hline \test{S} & Data una GUI con un elemento e più elementi funzionali, simulare un evento originato dall'elemento funzionale e verificare che questo abbia ripercussioni sulla bubble generica e sulla GUI & \\
%	\hline
%	\caption{Test di sistema per il framework}
%\end{longtable}

\subsubsection{Bubble To-do list}

\begin{longtable}{|c|P{9cm}|c|}
	\hline \multicolumn{1}{|l|}{\textbf{Codice}} &  \multicolumn{1}{l|}{\textbf{Descrizione}} & \multicolumn{1}{l|}{\textbf{Requisito}} \\ 
	\endfirsthead
	\hline \test{S} & Verificare che la bubble To-do list permetta all'utente la creazione di liste & \ref{AdR-L17} \\
	\hline \test{S} & Verificare che l'utente possa invocare il comando di creazione della to-do list & \ref{AdR-L71} \\
	\hline \test{S} & Verificare che la bubble carichi un form per l'inserimento delle informazioni & \ref{AdR-L95} \\
	\hline \test{S} & Verificare che l'utente possa inserire le informazioni per la creazione della to-do list all’interno del form & \ref{AdR-L72} \\
	\hline \test{S} & Verificare che l'utente possa confermare le informazioni inserite visualizzandole in un riepilogo & \ref{AdR-L73} \\
	\hline \test{S} & Verificare che la bubble To-do list permetta l’inserimento di nuovi elementi nella lista tramite un form & \ref{AdR-L18} \\
	\hline \test{S} & Verificare che una volta aggiunti nuovi elementi la bubble li mostri nella to-do list & \ref{AdR-L74} \\
	\hline \test{S} & Verificare che la bubble To-do list permetta di segnare come completati gli elementi della lista & \ref{AdR-L19} \\
	\hline \test{S} & Verificare che una volta segnato come completato un elemento della to-do list questo venga disabilitato e reso non modificabile & \ref{AdR-L75} \\
	\hline \test{S} & Verificare che la To-do list permetta di settare un reminder come notifica statica & \ref{AdR-L20} \\
	\hline \test{S} & Verificare che la notifica statica venga visualizzata una volta scaduto l'intervallo temporale impostato & \ref{AdR-L76} \\
	\hline
	\caption{Test di sistema per la bubble To-do list}
\end{longtable}

\subsubsection{Bubble \& eat}

\begin{longtable}{|c|P{9cm}|c|}
	\hline \multicolumn{1}{|l|}{\textbf{Codice}} & \multicolumn{1}{l|}{\textbf{Descrizione}} & \multicolumn{1}{l|}{\textbf{Requisito}} \\ 
	\endfirsthead
	
	\hline \test{S} & Verificare che la bubble per la ristorazione si interfacci con un database per il salvataggio delle informazioni & \ref{AdR-L21} \\
	\hline \test{S} & Verificare che la bubble per la ristorazione si interfacci con un database per il recupero delle informazioni & \ref{AdR-L61} \\
	\hline \test{S} & Verificare che la bubble per la ristorazione si interfacci con un database per aggiornare informazioni & \ref{AdR-L62} \\	
	\hline \test{S} & Verificare che i clienti per poter utilizzare la bubble possano inserire le proprie informazioni & \ref{AdR-L22} \\
	\hline \test{S} & Verificare che la bubble mostri un form per l'inserimento delle informazioni personali dell'utente & \ref{AdR-L96} \\
	\hline \test{S} & Verificare che l'utente possa inserire le informazioni nel form & \ref{AdR-L97} \\
	\hline \test{S} &Verificare che l'utente possa confermare le informazioni inserite visualizzandole nella bubble & \ref{AdR-L98} \\
	\hline \test{S} & Verificare che i clienti possano consultare il menù del ristorante, visualizzandolo nella bubble & \ref{AdR-L23} \\
	\hline \test{S} & Verificare che i clienti visualizzino i prezzi delle pietanze nel menù & \ref{AdR-L68} \\
	\hline \test{S} & Verificare che i clienti possano selezionare cibi e relative quantità per effettuare un ordine & \ref{AdR-L24} \\
	\hline \test{S} & Verificare che la quantità di default per un cibo non selezionato sia 0 & \ref{AdR-L69} \\
	\hline \test{S} & Verificare che la quantità di default per un cibo selezionato sia 1 & \ref{AdR-L77} \\
	\hline \test{S} & Verificare che non possa essere incrementata la quantità per un cibo non selezionato & \ref{AdR-L70} \\
	\hline \test{S} & Verificare che l'utente possa confermare di aver concluso la modifica cliccando su un pulsante di conferma & \ref{AdR-L99} \\
	\hline \test{S} & Verificare che la bubble mostri all'utente un resoconto dell'ordine prima della sua conferma definitiva & \ref{AdR-L100} \\
	\hline \test{S} & Verificare che l'utente Cuoco possa visualizzare la lista di piatti da preparare & \ref{AdR-L25} \\
	\hline \test{S} & Verificare che l'utente Cuoco possa marcare come completati i piatti già preparati tra quelli dalla lista di piatti da preparare  & \ref{AdR-L26} \\
	\hline \test{S} & Verificare che i piatti spuntati dal Cuoco vengono eliminati dalla lista dei piatti da preparare & \ref{AdR-L78} \\
	\hline \test{S} & Verificare che l'utente Responsabile Acquisti possa visualizzare la lista degli acquisti da effettuare & \ref{AdR-L27} \\
	\hline \test{S} & Verificare che l'utente Responsabile Acquisti possa spuntare i prodotti che ha acquistato dalla lista acquisti & \ref{AdR-L28} \\
	\hline \test{S} & Verificare che l'utente Direttore possa visualizzare il menù del ristorante all’interno della bubble & \ref{AdR-L104}  \\
	\hline \test{S} & Verificare che l'utente Direttore possa cambiare il menù del ristorante & \ref{AdR-L105} \\
	\hline \test{S} & Verificare che l'utente Direttore possa modificare le voci del menù ed i relativi prezzi all’interno della bubble & \ref{AdR-L29} \\
	\hline \test{S} & Verificare che l'utente Direttore possa aggiungere le voci del menù con relativi prezzi tramite un form & \ref{AdR-L79} \\
	\hline \test{S} & Verificare che l'utente Direttore possa selezionare l'opzione per aggiungere elemento & \ref{AdR-L108} \\
	\hline \test{S} & Verificare che l'utente Direttore possa rimuovere le voci del menù & \ref{AdR-L80} \\
	\hline \test{S} & Verificare che l'utente Direttore possa selezionare le voci del menù & \ref{AdR-L106} \\
	\hline \test{S} & Verificare che l'utente Direttore possa visualizzare la lista dei piatti da preparare all’interno della bubble & \ref{AdR-L30} \\
	\hline \test{S} & Verificare che l'utente Direttore possa visualizzare la lista degli acquisti da effettuare all’interno della bubble & \ref{AdR-L31} \\
	\hline \test{S} & Verificare che l'utente Direttore possa eliminare voci dalla lista dei piatti da preparare & \ref{AdR-L32} \\
	\hline \test{S} & Verificare che l'utente Direttore possa selezionare ordini nella lista dei piatti da preparare & \ref{AdR-L101} \\
	\hline \test{S} & Verificare che l'utente Direttore possa selezionare l'opzione per rimuovere gli elementi selezionati & \ref{AdR-L102} \\
	\hline \test{S} & Verificare che l'utente Direttore possa selezionare l'opzione per modificare gli elementi selezionati & \ref{AdR-L107} \\
	\hline \test{S} & Verificare che l'utente Direttore possa confermare di voler effettuare l'operazione tramite un pulsante & \ref{AdR-L103} \\
	\hline \test{S} & Verificare che l'utente Direttore possa aggiungere ingredienti alla lista degli acquisti da effettuare & \ref{AdR-L53}\\
	\hline \test{S} & Verificare che l'utente Direttore possa eliminare ingredienti dalla lista degli acquisti da effettuare & \ref{AdR-L65} \\
	\hline \test{S} & Verificare che l'utente Direttore possa selezionare elementi dalla lista degli acquisti da effettuare & \ref{AdR-L109} \\
	\hline \test{S} & Verificare che l'utente Fattorino possa visualizzare la lista delle consegne da effettuare all’interno della bubble & \ref{AdR-L50}  \\
	\hline \test{S} & Verificare che l'utente Fattorino possa selezionare la consegna che intende effettuare & \ref{AdR-L51} \\
	\hline \test{S} & Verificare che l'utente Fattorino possa confermare la consegna che ha effettuato & \ref{AdR-L52} \\
	\hline
	\caption{Test di sistema per la bubble Bubble \& eat}
\end{longtable}
