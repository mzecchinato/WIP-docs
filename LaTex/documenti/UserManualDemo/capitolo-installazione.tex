\section{To-do List Installation's guide}

\subsection{System Requirements}
The bubble To-do list can be used in any Browser that supports HTML5. To be able to install the To-do list Bubble, npm v4.2.0 and nodejs v7.10.0 are required. The informations on how to install them can be found by following these links:
\begin{itemize}
	\item Node.js: \url{https://nodejs.org/en/download/package-manager/};
	\item Npm: \url{https://docs.npmjs.com/cli/install}.
\end{itemize}

\subsection{Installation}
\subsubsection{Windows}

To install the bubble To-do list on a Windows operating system please follow this procedure:
\begin{itemize}
	\item open the command propt, or any command-line interface program of your choice;
	\item navigate to the server directory through the command propt;
	\item launch the command ``npm i \&\& npm start'';
	\item navigate to the client directory through the command propt;
	\item launch the command ``npm i \&\& npm start'';
	\item when the client and server have finished installing, a new browser tab will open where it will be possible to open the bubble.
\end{itemize}

\subsubsection{Linux}
To install the bubble To-do list on a Linux operating system please follow this procedure:
\begin{itemize}
	\item open the terminal, or any command-line interface program of your choice;
	\item navigate to the application directory through the terminal; 
	\item start the installation's script by typing ``./start.sh'' in the terminal and then click enter;
	\item when the application has finished installing, a new browser tab will open where it will be possible to open the bubble.
\end{itemize}

In the event of failure of the installation process please follow this alternative procedure for installing the application:
\begin{itemize}
	\item open the terminal, or any command-line interface program of your choice;
	\item navigate to the server directory through the terminal;
	\item launch the command ``npm i \&\& npm start'';
	\item navigate to the client directory through the terminal;
	\item launch the command ``npm i \&\& npm start'';
	\item when client and server have finished installing, a new browser tag will open where it will be possible to open the bubble.
\end{itemize}

\section{\DemoName{} Installation's guide}

\subsection{ \DemoName{} System Requirements}
\DemoName{} can be used in any Browser that supports HTML5. 
To be able to install \DemoName{} Bubble, npm v4.2.0 and nodejs v7.10.0 are required. The informations on how to install them can be found by following these links:
\begin{itemize}
	\item Node.js: \url{https://nodejs.org/en/download/package-manager/};
	\item Npm: \url{https://docs.npmjs.com/cli/install}.
\end{itemize}

\subsection{Installation}
\subsubsection{Windows}

To install \DemoName{} on a Windows operating system please follow this procedure:
\begin{itemize}
	\item open the command propt, or any command-line interface program of your choice;
	\item navigate to the server directory through the command propt;
	\item launch the command ``npm i \&\& npm start'';
	\item navigate to the client directory through the command propt;
	\item launch the command ``npm i \&\& npm start'';
	\item when the client and server have finished installing, a new browser tab will open where it will be possible to open the bubble.
\end{itemize}

\subsubsection{Linux}
To install \DemoName{} on a Linux operating system please follow this procedure:
\begin{itemize}
	\item open the terminal, or any command-line interface program of your choice;
	\item navigate to the application directory through the terminal; 
	\item start the installation's script by typing ``./start.sh'' in the terminal and then click enter;
	\item when the application has finished installing, a new browser tab will open where it will be possible to open the bubble.
\end{itemize}

In the event of failure of the installation process please follow this alternative procedure for installing the application:
\begin{itemize}
	\item open the terminal, or any command-line interface program of your choice;
	\item navigate to the server directory through the terminal;
	\item launch the command ``npm i \&\& npm start'';
	\item navigate to the client directory through the terminal;
	\item launch the command ``npm i \&\& npm start'';
	\item when client and server have finished installing, a new browser tag will open where it will be possible to open the bubble.
\end{itemize}
