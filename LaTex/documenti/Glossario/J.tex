\letteraGlossario{J}
\definizione{Java}
linguaggio di programmazione orientato agli oggetti, progettato per essere indipendente dalla piattaforma di esecuzione utilizzando l'implementazione di un processore virtuale detto \glossario{JVM}.

\definizione{JavaScript}
Linguaggio di scripting orientato agli oggetti e agli eventi, utilizzato nella programmazione web lato client per la creazione di effetti dinamici in siti e applicazioni web.

\definizione{JetBrains}
JetBrains è un'azienda di sviluppo software i cui strumenti hanno come mercato gli sviluppatori software e i project manager.\\
\url{https://www.jetbrains.com/}

\definizione{Jira}
Jira è prodotto proprietario di issue tracking, sviluppato da Atlassian. Fornisce tracciamento dei bug, delle segnalazioni e funzioni per la gestione di progetto.\\
\url{https://www.atlassian.com/software/jira}

\definizione{JSDoc}
JSDoc è un \glossario{linguaggio di markup} usato per annotare file di codice sorgente \glossario{JavaScript}. Usando commenti contenenti JSDoc, i programmatori possono aggiungere documentazione descrivendo l’interfaccia di programmazione del codice dell’applicazione. Questo viene quindi processato per produrre documentazione in formati accessibili come \glossario{HTML}.\\
\url{http://usejsdoc.org/}

\definizione{JSON}
Acronimo di JavaScript Object Notation, è un formato usato nell'interscambio di dati fra applicazioni client-server.

\definizione{JVM}
Acronimo di Java Virtual Machine è il componente della piattaforma Java che esegue i programmi tradotti in bytecode dopo una prima compilazione.
\clearpage