\letteraGlossario{O}

\definizione{OOP}
OOP è l'acronimo di \virgolette{Object Oriented Programming} (tradotto in programmazione orientata agli oggetti) è un paradigma di programmazione che permette di definire oggetti software in grado di interagire gli uni con gli altri attraverso lo scambio di messaggi.\\
Un ambito che più di altri riesce a sfruttare i vantaggi della programmazione ad oggetti è quello delle interfacce grafiche.

\definizione{Open Project}
OpenProject è uno strumento di gestione di progetto \glossario{open source}. Nella versione gratuita è necessario utilizzarlo su un proprio server.\\
\url{https://www.openproject.org/}

\definizione{Open source}
Accostato ad un software sta ad indicare che il codice sorgente dello stesso è pubblico, favorendone lo studio, modifiche ed estensioni da parte di programmatori indipendenti.

\definizione{Overleaf}
Overleaf è un editor online di \glossario{LaTeX} che permette il lavoro collaborativo di più persone. Offre una versione gratuita con un limite allo spazio disponibile in \glossario{cloud} per il salvataggio dei documenti.\\
\url{https://www.overleaf.com/}
\clearpage
