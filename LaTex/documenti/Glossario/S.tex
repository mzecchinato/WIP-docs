\letteraGlossario{S}
\definizione{Sass}
Sass (Syntactically Awesome StyleSheets) è un'estensione del linguaggio \glossario{CSS} che permette di utilizzare variabili, creare funzioni e organizzare il foglio di stile in più file. Il linguaggio Sass si basa sul concetto di preprocessore \glossario{CSS}, il quale serve a definire fogli di stile con una forma più semplice, completa e potente rispetto al \glossario{CSS} e a generare file \glossario{CSS} ottimizzati aggregando, anche in modo complesso, le strutture definite.\\
\url{http://sass-lang.com/}

\definizione{SCSS}
SCSS corrisponde alla versione 3 di \glossario{Sass}. Introduce la piena compatibilità con tutte le versioni di \glossario{CSS}.

\definizione{SDK}
Un Software Development Kit indica genericamente un insieme di strumenti per lo sviluppo e la documentazione di software.

\definizione{Skype}
Software proprietario freeware di instant messaging e VoIP. Con esso sono possibili le videochiamate e lo scambio di messaggi testuali o di file.\
\url{https://www.skype.com/it/}

\definizione{Slack}
Piattaforma di messaggistica diffusa in ambito aziendale, permette la creazione di canali dedicati e la suddivisione in organizzazioni. 

\definizione{Socket}
Un socket è un'astrazione software per la trasmissione e la ricezione di dati attraverso una rete. È il punto in cui il codice applicativo di un processo accede al canale di comunicazione per mezzo di una porta, ottenendo una comunicazione tra processi che lavorano su due macchine fisicamente separate.

\definizione{Software-as-a-Service (SaaS)}
Software-as-a-Service (SaaS) (Software come servizio in italiano) è un modello di distribuzione del software applicativo dove un produttore di software sviluppa, opera (direttamente o tramite terze parti) e gestisce un'applicazione web che mette a disposizione dei propri clienti via Internet. Si tratta di un servizio di \glossario{cloud} computing.

\definizione{Software Libero}
Il \virgolette{Software libero} è software che rispetta la libertà degli utenti e la comunità. In breve, significa che gli utenti hanno la libertà di eseguire, copiare, distribuire, studiare, modificare e migliorare il software. Quindi è una questione di libertà, non di prezzo.
Un programma è software libero se gli utenti del programma godono delle quattro libertà fondamentali:
\begin{itemize}
    \item libertà di eseguire il programma come si desidera, per qualsiasi scopo (libertà 0);
    \item libertà di studiare come funziona il programma e di modificarlo in modo da adattarlo alle proprie necessità (libertà 1). L'accesso al codice sorgente ne è un prerequisito;
    \item libertà di ridistribuire copie in modo da aiutare il prossimo (libertà 2);
    \item libertà di migliorare il programma e distribuirne pubblicamente i miglioramenti da voi apportati (e le vostre versioni modificate in genere), in modo tale che tutta la comunità ne tragga beneficio (libertà 3). L'accesso al codice sorgente ne è un prerequisito.
\end{itemize}
Maggiori informazioni si trovano all'indirizzo \url{https://www.gnu.org/philosophy/free-sw.it.html} .

\definizione{SQL}
Acronimo di Structured Query Language è un linguaggio standardizzato per database che utilizzano il modello relazionale.

\definizione{Stakeholder}
Con il termine stakeholder (o portatore di interesse) si indica genericamente un soggetto (o un gruppo di soggetti) influente nei confronti di un'iniziativa economica, che sia un'azienda o un progetto.
Fanno, ad esempio, parte di questo insieme: clienti, fornitori, finanziatori come banche e azionisti (o shareholder), collaboratori, dipendenti ma anche gruppi di interesse locali o gruppi di interesse esterni, come residenti di aree limitrofe all'azienda e istituzioni statali relative all'amministrazione locale.

\definizione{Strumento di build}
Uno strumento di build è uno strumento usato per realizzare una nuova versione di un programma. Per esempio, \textit{make} è un popolare strumento di build open source che usa \textit{makefile}, un altro strumento di build, per assicurare che i file sorgente che sono stati aggiornati (e i file che dipendono da questi) vengano compilati in una nuova versione (build) di un programma.

\definizione{Stub}
Componente passiva fittizia per simulare una parte del sistema non oggetto di test. \`{E} il duale del driver. Rappresenta una o più componenti necessarie per l’avanzamento dei test del programma ma non ancora implementate.

\definizione{SVG}
Acronimo di Scalable Vector Graphics, indica una tecnologia in grado di visualizzare oggetti di grafica vettoriale e, pertanto, di gestire immagini scalabili dimensionalmente.

\definizione{SVN}
Apache Subversion (noto anche come svn, che è il nome del suo client a riga di comando) è un sistema di versionamento e revisione per software, distribuito gratuitamente sotto licenza Apache. \`{E} stato progettato con lo scopo di essere il naturale successore di CVS, oramai considerato superato.\\
\url{https://subversion.apache.org/}
\clearpage
