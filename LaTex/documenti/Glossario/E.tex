\letteraGlossario{E}
\definizione{ECMAScript}
ECMAScript (o ES) è un linguaggio di programmazione standardizzato e mantenuto da Ecma International nell'ECMA-262 ed ISO/IEC 16262. Le implementazioni più conosciute di questo linguaggio sono \glossario{JavaScript}, JScript e ActionScript che sono entrati largamente in uso, inizialmente, come linguaggi client-side nel web development.
La versione stabile più recente dello standard è ECMAScript 2016 (\url{www.ecmascript.org/}).

\definizione{Elemento}
Elemento del \glossario{framework} generico che può essere un \glossario{elemento grafico}, un \glossario{elemento funzionale}, un \glossario{elemento di input}, un \glossario{elemento di output} oppure un insieme di questi.

\definizione{Elemento funzionale}
Elemento del \glossario{framework} con una determinata funzionalità che ha lo scopo di realizzare le operazioni esposte solitamente all'utente attraverso l’interfaccia grafica (\glossario{elemento grafico}).

\definizione{Elemento grafico}
Elemento dell’interfaccia grafica del \glossario{framework} con lo scopo di poter visualizzare dei dati o di fornire possibilità di interazione all’utente.

\definizione{Elemento di input}
\glossario{Elemento grafico} che può essere aggiunto alla \glossario{bubble generica} per svolgerne le funzionalità di input. Ogni elemento di input verrà assegnato ad una variabile all’interno della \glossario{bubble memory}. Gli elementi di input possono essere di vari tipi.

\definizione{Elemento di output}
\glossario{Elemento grafico} che può essere aggiunto alla \glossario{bubble generica} per svolgerne le funzionalità di output. Ogni elemento di output verrà assegnato ad una variabile all’interno della \glossario{bubble memory}. Gli elementi di output possono essere di vari tipi.

\definizione{Ereditarietà}
L'ereditarietà è uno dei concetti fondamentali nel paradigma di programmazione a oggetti.\\
Il modo in cui i linguaggi di programmazione gestiscono le relazioni di ereditarietà consegue dal significato dato all'ereditarietà come relazione is-a. Una classe B dichiarata sottoclasse di un'altra classe A:
\begin{itemize}
    \item eredita (ha implicitamente) tutte le variabili di istanza e tutti i metodi di A;
    \item può avere variabili o metodi aggiuntivi;
    \item può ridefinire i metodi ereditati da A attraverso l'overriding, in modo che essi eseguano la stessa operazione concettuale in un modo specializzato.
\end{itemize}

\definizione{ESLint}
Strumento di linting per \glossario{JavaScript}.\\
\url{http://eslint.org/}
\clearpage
