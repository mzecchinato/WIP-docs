\section{Lista di controllo}
Durante l'analisi statica tramite walkthrough effettuata sui documenti, sono stati rilevati con maggior frequenza i seguenti errori:
\begin{itemize}
	\item gli acronimi vengono scritti in minuscolo;
	\item la lettera \virgolette{è} accentata viene scritta \virgolette{É} invece che \virgolette{È};
	\item vengono inseriti spazi multipli tra le parole;
	\item non vengono usati i comandi creati per alcuni termini, tra cui \ProjectName{};
	\item vari errori grammaticali e di sintassi, tra cui l'inversione delle lettere nelle parole;
	\item viene usato il futuro invece che il presente, contro quanto riportato nelle \NormeDiProgetto{};
	\item i nomi delle bubble vengono scritti diversamente da quanto concordato (To-do List, todo list, invece che To-do list);
	\item il termine JavaScript viene spesso scritto Javascript;
	\item il termine Bubble \& eat viene scritto non correttamente Bubble \& Eat;
	\item nei documenti in cui è stato deciso di utilizzare le traduzioni inglesi di alcuni termini, sono presenti ancora utilizzi in italiano;
	\item i termini inglesi che indicano pluralità vengono scritti nella loro forma inglese, contro quanto riportato nelle \NormeDiProgetto{};
	\item non vengono usate le parentesi graffe al termine dei comandi \LaTeX, ottenendo così di fatto parole non intervallate dallo spazio;
	\item il termine database viene spesso mal digitato \textit{databse};
	\item vengono omessi i punti e virgola al termine delle frasi degli elenchi;
	\item la parola menu viene scritta nella versione menù, contrariamente a quanto deciso internamente al gruppo.
\end{itemize}