\section{Riunione}
\subsection{Ordine del Giorno}
\begin{itemize}
	\item Canali di comunicazione
	\item Docker, Heroku e MongoLab
	\item Bubbles
	\item SDK e Bubbles
	\item Angular2 e React
	\item Repository e MIT License
	\item Documentazione e Lingua
	\item Comunicazione e Riunioni interni al gruppo \GroupName{}
	\item Incontro con \Proponente{}	
\end{itemize}

\subsection{Dialogo con \Proponente}
\subsubsection{Canali di comunicazione}
Sono già disponibili canali dedicati ai gruppi impegnati nella realizzazione del capitolato C5 sulla piattaforma di comunicazione \glossario{Slack}. Questo strumento permette di comunicare in maniera più immediata rispetto alle \email{} e favorisce l'interazione fra i diversi gruppi. VE_2016-12-8_D1: si conferma l'utilizzo della piattaforma di comunicazione Slack.

\subsubsection{Docker, Heroku e MongoLab}
\glossario{Heroku} è consigliato data la sua semplicità rispetto a \glossario{Docker}.\\ 
Per interfacciarsi con i database è stato consigliato di vedere MongoLab.
VE_2016-12-8_D2: dato l'esito della discussione l'utilizzo di MongoDB ed Heroku è confermato per facilità d'uso e necessità di caricamento della demo su tale piattaforma. 

\subsubsection{Bubbles}
Per la creazione delle \glossario{bubble} sono presenti è stata lasciata completa libertà. Sono forniti alcuni esempi concettuali, non vincolanti rispetto al prodotto finale.

\subsubsection{SDK e Bubbles}
È stato chiarito che la prima parte del progetto riguarda la libreria per la creazione delle bubble, mentre la seconda è l'utilizzo di tale libreria per la creazione delle bubble. L'\glossario{SDK} deve quindi contenere primitive per la creazione delle bubble e alcuni template esemplificativi, le bubble della demo stesse, che utilizzino le primitive.\\
Per mostrare le vere potenzialità dell'SDK, bisogna considerare gli utilizzi non banali all'interno della chat, come l'auto-aggiornamento dei dati.

\subsubsection{Angular2 e React}
Vi è uno scarso utilizzo di Blaze all'interno di \glossario{Meteor}, mentre è ampia la diffusione di React, che rimane dunque il più consigliato. Angular2 è probabilmente in evoluzione e pertanto potrebbe esserci un problema di compatibilità nell'aggiornamento.
VE_2016-12-8_D3: per le considerazioni effettuate insieme a \Proponente React rimane la libreria selezionata.

\subsubsection{Repository e MIT License}
Il repository è gestito e creato dal gruppo \GroupName{} con licenza \glossario{MIT} e una dichiarazione di proprietà del \Proponente{}.

\subsubsection{Documentazione e Lingua}
La documentazione del software ed i commit nel repository vanno effettuati in lingua inglese. I commit devono essere autoesplicativi.

\subsubsection{Comunicazione e Riunioni interni al gruppo \GroupName{}}
Sono consigliati la comunicazione costante tra i componenti e frequenti incontri interni al gruppo \GroupName{}. Alcune considerazioni di \Proponente{} riguardo alla collaborazione sono: 
\begin{itemize}
	\item comunicazione costante;
	\item completa trasparenza;
	\item aiuto fra membri del gruppo;
	\item coordinazione rispetto agli impegni;
	\item imposizione di regole.
\end{itemize}

\subsubsection{Incontro con \Proponente}
È stato fissato entro la fine dell'anno 2016 un incontro con \Proponente{}, probabilmente durante il periodo Natalizio.

\clearpage
