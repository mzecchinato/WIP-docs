\section{Riunione}
\subsection{Ordine del Giorno}
\begin{itemize}
	\item Resoconto situazione
	\item Resoconto disponibilità oraria
	\item Discussione comportamento fattorino
	\item Discussione parti bubble contenute nel framework 
	\item Immagini progetto Visual Paradigm
\end{itemize}

\subsection{Discussione e decisioni}

\subsubsection{Resoconto situazione}
La struttura e la stesura delle scelte progettuali sono state completate. La visione d'insieme rimane generica, ma è stata confermata la scelta dei design pattern. Nel complesso sono state confermate le tecnologie da utilizzare.

\subsubsection{Resoconto disponibilità oraria}
Il gruppo rimane attivo, nonostante le lezioni e gli impegni lavorativi rendano difficile la conclusione anticipata dei lavori e la collaborazione negli stessi orari, come invece era successo durante il periodo antecedente la prima revisione. È stato creato un poll per verificare la disponibilità giornaliera di ogni componente, rispettivamente durante il mattino e il pomeriggio, al fine di mettere tutto il gruppo a conoscenza di ciò.

\subsubsection{Discussione comportamento fattorino}
VI\_2017-03-2\_D1: Per rendere la demo il più utile e affidabile possibile per una situazione di utilizzo reale, si è deciso di fare in modo che il fattorino, per le consegne del ristorante che possiede la Bubble \& Eat, possa prenotare le consegne, in modo da poter gestire il proprio tempo ed il proprio percorso più efficientemente. In questo modo è inoltre possibile prevenire l'impostazione di un fittizio stato di consegna, nel caso in cui il fattorino stia portando a termine una data consegna e voglia prenderne in carico altre. Viene quindi così impedito a un fattorino di prenotare consegne che non può portare a termine, bloccando di fatto la lista per eventuali altri colleghi.

\subsubsection{Discussione parti bubble contenute nel framework}
VI\_2017-03-2\_D2: Si è deciso di considerare le parte delle \glossario{bubble} già presenti nel \glossario{framework} come direttamente collegate al \glossario{framework} e non di esclusiva competenza di ciascuna \glossario{bubble}, con conseguenze nelle descrizioni e nei grafici relativi ad esse.

\subsubsection{Immagini progetto Visual Paradigm}
VI\_2017-03-2\_D3: Si è deciso di gestire le immagini all'interno di un progetto unico di \glossario{Visual Paradigm} in modo da poter controllarne esportazione e modifica in modo più semplice e diretto.

\clearpage
