%%%%%%%%%%%%%%
%  COSTANTI  %
%%%%%%%%%%%%%%

% In questa prima parte vanno definite le 'costanti' utilizzate soltanto da questo documento.
% Devono iniziare con una lettera maiuscola per distinguersi dalle funzioni.

\newcommand{\DocTitle}{Definizione di Prodotto}
\newcommand{\DocVersion}{\VersioneDP{}}

\newcommand{\DocRedazione}{\lorenzo \\ \bea \\ \nick \\ \alice \\ \tommy \\ \marco \\ \mattia}
\newcommand{\DocVerifica}{\bea \\ \alice}
\newcommand{\DocApprovazione}{\tommy}

\newcommand{\DocUso}{Esterno}
\newcommand{\DocDistribuzione}{
	\Committente{} \\
	Gruppo \GroupName{}
}

% La descrizione del documento
\newcommand{\DocDescription}{
Documento relativo alla descrizione del prodotto e dei componenti che costituiscono il progetto svolto dal gruppo \GroupName{}.
}

%%%%%%%%%%%%%%
%  FUNZIONI  %
%%%%%%%%%%%%%%

% In questa seconda parte vanno definite le 'funzioni' utilizzate soltanto da questo documento.
% Setter e getter per il nome::delle::classi::che::è::infinito::da::scrivere
\newcommand{\class}{}
\newcommand{\setclass}[1]{
	\renewcommand{\class}{#1}
}

% Comando per i riferimenti alle classi
\newcommand{\cref}[1]{\hyperref[#1]{\texttt{#1}}}
\newcommand{\coderef}[1]{\hyperref[#1]{\code{#1}}}

% Change symbol for second level lists
\renewcommand{\labelitemii}{$\circ$}

\definecolor{green}{RGB}{0,102,0}
\definecolor{blue}{RGB}{0,43,128}
\newcommand{\method}[1]{\textcolor{blue}{\code{#1}}}
\newcommand{\field}[1]{\textcolor{green}{\code{#1}}}
\newcommand{\param}[1]{\textcolor{black}{\code{#1}}}

\usepackage{xr}
\externaldocument[AdR-]{../AnalisiDeiRequisiti/capitolo-requisiti}