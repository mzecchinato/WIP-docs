\section{Requisiti}
Vengono ora presentati i requisiti emersi durante l'analisi del capitolato e di ogni caso d'uso e i requisiti discussi nelle riunioni interne e con il Proponente.
Si è deciso di inserire i requisiti in una tabella dei requisiti per permettere una consultazione agevole degli stessi.
La tabella dei requisiti presenta i requisiti fino al massimo livello di dettaglio insieme alle loro caratteristiche, in particolare ne specifica:
\begin{itemize}
	\item codice (come stabilito nelle \NormeDiProgetto{});
	\item categoria di appartenenza fra:
	\begin{itemize}
		\item obbligatorio, per i requisiti irrinunciabili per un qualsiasi \glossario{stakeholder};
		\item desiderabile, per i requisiti non strettamente necessari, ma che offrono un valore aggiunto riconoscibile;
		\item opzionale, per i requisiti relativamente utili o contrattabili in seguito.
	\end{itemize}
	\item una descrizione esaustiva del requisito;
	\item le fonti dal quale il requisito ha avuto origine, sia essa l'analisi diretta del capitolato oppure il dialogo con i Proponenti e/o in base alle necessità architetturali ed implementative del progetto individuate tramite casi d'uso;	
\end{itemize}