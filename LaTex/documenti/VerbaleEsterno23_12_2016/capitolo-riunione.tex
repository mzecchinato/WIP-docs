\section{Riunione}
\subsection{Ordine del Giorno}
\begin{itemize}
	\item Documentazione e manualistica
	\item Target per le \glossario{bubble}
	\item \glossario{API}
	\item Demo
	\item Desktop e mobile
	\item Proposte e suggerimenti
\end{itemize}

\subsection{Dialogo con \Proponente}
\subsubsection{Documentazione e Manualistica}
La documentazione rivolta all'utente finale, come il manuale utente, deve  essere prodotta in forma di documento non verboso o come pagina web e deve essere esplicativa delle funzionalità del prodotto. 
Essa sarà inclusa nel \glossario{repository} di progetto e dovrà essere prodotta in lingua inglese.

\subsubsection{Ruoli}
Gli utilizzatori delle \glossario{bubble} possono avere diversi ruoli. Si può così individuare una distinzione fra chi gestisce la bubble e ne determina il funzionamento e gli utilizzatori.
Grazie a questa distinzione è possibile considerare l'uso di bubble per comunicazioni formali. 
L'amministratore si può così porre ad un livello di controllo più alto dando agli utilizzatori la possibilità di usare le bubble in contesti di rapporti lavorativi e anche di servizi e commercio.
VE_2016-12-23_D1: per la progettazione delle bubble vengono considerate attività che richiedano la gestione di ruoli differenti.

\subsubsection{Idee}
I ruoli e l'integrazione di servizi di terze parti sono stati ritenuti apprezzabili.
\`{E} stato consigliato di dialogare con personale amministrativo che abbia rapporti con il pubblico, per esempio la segreteria universitaria, e di considerare i rapporti fra chi offre servizi e chi li riceve, per considerare la realizzazione di strumenti che permettano a questi rapporti di avvenire tramite applicazioni di messaggistica.  

\subsubsection{Target delle bubble}
Il target è composto dagli utenti di \glossario{Rocket.Chat}.\\
Non ci sono preferenze per da parte di \Proponente{} sul tipo di bubble o sul target delle bubble da produrre. 

\subsubsection{API esterne e APIKey}
Per quanto riguarda l'utilizzo di \glossario{API} esterne che richiedano autenticazione o key si rimanda l'onere all'amministratore della bubble.
VE_2016-12-23_D2: in sede di progettazione la gestione delle API esterne lascia all'amministratore l'onere di gestione dell'autenticazione per le API che lo richiedano.

\subsubsection{Configurazione}
La parte di configurazione della \glossario{bubble} va astratta dai concetti di input, output, logica e stato.

\subsubsection{Demo}
La dimostrazione delle potenzialità del \glossario{framework} è parte integrante del capitolato.\\
Va posto l'accento sulla direzione da prendere e non sui dettagli tecnici in modo da avere una visuale sulle \glossario{bubble} da implementare e includere nella demo. Si consiglia di prendere in considerazione due o tre \glossario{bubble}.
Nella dimostrazione deve essere presente la distinzione fra ruoli, mostrando quindi una situazione di interazione di utenti a parità di livello e una situazione di customer service, se presenti nelle bolle considerate. 

\subsubsection{Mobile}
Quando è stato offerto il capitolato non è stata considerata la tecnologia con cui \glossario{Rocket.Chat} è stato realizzato per la parte mobile. Si è considerato che durante il processo di creazione è probabile siano state usate varie tecnologie fra le quali \glossario{Cordova} per la generazione automatica del codice per mobile. Sotto queste ipotesi è stato considerato di realizzare la compatibilità delle \glossario{bubble} sotto forma di web view, oppure considerare di sviluppare un \glossario{SDK} specifico per mobile a scelta fra un solo sistema: \glossario{iOS} o \glossario{Android}.  
Emerge l'inutilità di bubble specifiche per mobile in un contesto di utilizzo desktop.
VE_2016-12-23_D3: si assegna quindi priorità alla progettazione desktop.

\clearpage
