\section{Riunione}
\subsection{Ordine del Giorno}
\begin{itemize}
	\item Canali di comunicazione
	\item Riunioni
	\item Repository per tecnologie utilizzate
	\item Struttura Documenti
	\item Definizione Ruoli
	\item Ambiente di lavoro
\end{itemize}

\subsection{Discussione e decisioni}
\subsubsection{Canali di comunicazione}
VI\_2016-12-9\_D1: Sono stati provati e decisi i  vari canali di comunicazioni da utilizzare nell'arco del progetto.
I canali scelti sono stati:
\begin{itemize}
	\item \textbf{GMail}: per le comunicazioni con \CommittenteInline{}, attraverso un account di gruppo generato appositamente (\GroupEmail{});
	\item \textbf{Slack}: per le comunicazioni con \Proponente{}, scelta consigliata dallo stesso;
	\item \textbf{Discord}: per la collaborazione durante il progetto in quanto strutturato per chat vocali non funzionanti anche in situazioni non ottimali;
	\item \textbf{Telegram}: per le comunicazioni rapida all'interno del gruppo, la scelta è stata immediata visto che ogni membro del gruppo ne possiede già un account.
\end{itemize}
Sono stati scartati Skype e Hangouts, in quanto nelle prove abbiamo riscontrato vari problemi nell'utilizzo.

\subsubsection{Riunioni}
VI\_2016-12-9\_D2: La cadenza delle riunioni è stata decisa per una a settimana via chat vocale, ed una ogni due settimane in un incontro di persona.
La distanza tra i componenti e i vari impegni non permettono incontri fisici troppo frequenti, che causerebbero una notevole perdita di tempo negli spostamenti, si è deciso di limitarli a quando possibile o quando strettamente necessario.

\subsubsection{Repository per tecnologie utilizzate}
Sono state create le repository per il lavoro collaborativo e i cui tenere il materiale sviluppato durante il progetto.

\subsubsection{Struttura Documenti}
VI\_2016-12-9\_D3: La struttura dei documenti è stata decisa in modo collaborativo, e si effettuerà una prima stesura degli stessi per poi discutere in seguito sui possibili miglioramenti.

\subsubsection{Definizione Ruoli}
VI\_2016-12-9\_D4: Sono stati discussi e definiti i ruoli e i periodi che compongono il progetto.
Alla fine della discussione è stata redatta una bozza dell'organizzazione dei ruoli che sarà presentata nel \PianoDiProgetto{}.

\subsubsection{Ambiente di lavoro}
VI\_2016-12-9\_D5: Sono state prese decisione per rendere l'ambiente di sviluppo più unificato possibile, nonostante i membri del gruppo lavorino al momento su sistemi operativi differenti.
Si è deciso di condividere i malfunzionamenti per cercare una soluzione comune che non si ripercuota negli altri ambienti di lavoro.

\clearpage
