\section{Altri capitolati}

\subsection{Capitolato C1 - APIM: An API Market Platform}
\subsubsection{Descrizione}
Il capitolato propone la realizzazione di un software volto a gestire un mercato virtuale di API, che i venditori possono offrire attraverso varie modalità di vendita e i compratori possono acquistare per utilizzarle direttamente o integrarle insieme ad altre.\\
Il mercato dovrà essere funzionante ed utilizzabile, quindi prevedere oltre alle funzionalità dirette all'utente anche una serie di funzioni per la sicurezza e l'affidabilità del sistema.

\subsubsection{Valutazione}
Il prodotto risultante dal capitolato permetterà ad ogni sviluppatore di inserire le proprie API all'interno del mercato, e gestirne la vendita e l'utilizzo da parte degli altri utenti.\\
La presenza di più gruppi avrebbe fatto variare i requisiti, non permettendo un'analisi definitiva fin da subito.

\subsection{Capitolato C2 - AtAVi: Accoglienza tramite Assistente Virtuale}
\subsubsection{Descrizione}
Il capitolato si prefigge l'obiettivo di realizzare un sistema di accoglienza virtuale che sarà inserito negli uffici di zero12\footnote{\url{www.zero12.it}}, con lo scopo di ricevere i clienti ed informare gli impiegati, tramite servizi multimediali e di chat, della presenza di tali ospiti.

\subsubsection{Valutazione}
Il prodotto sfrutterà tecnologie che al giorno d'oggi sono sempre più disponibili e di tendenza, ma il prodotto risultante da questo capitolato verrà utilizzato direttamente nell'azienda proponente e gli utilizzatori saranno clienti che effettueranno visite fisiche negli uffici di zero12.\\
Data l'affinità dei componenti del gruppo con le idee per l'implementazione di tecnologie \glossario{open source} ad ampia diffusione si è deciso di valutare altri capitolati.

\subsection{Capitolato C3 -  DeGeOP: A Designer and Geo-localizer Web App for Organizational Plants}
\subsubsection{Descrizione}
Il progetto offre alle aziende un sistema di valutazione del processo produttivo e dei rischi collegati ad esso. L'algoritmo prende in input nodi rappresentanti le parti del processo produttivo, selezionati rispetto ad una localizzazione geografica, e fornisce in output una rappresentazione tramite grafo, dotato di informazioni riguardanti le valutazioni dei rischi aziendali e ambientali.\\
Il gruppo dovrà realizzare un'interfaccia, utilizzando le API, il \glossario{front-end} e il \glossario{back-end} messi a disposizione, che permetta, attraverso una mappa, l'inserimento e la successiva visualizzazione dei dati elaborati dall'algoritmo. Tale interfaccia deve essere utilizzabile tramite una web app disponibile online e offline, compatibile con il mondo mobile dei tablet e degli smartphone, quindi gestire le gesture tipiche dei dispositivi touchscreen.

\subsubsection{Valutazione}
Considerata la presenza di interfacce di front-end, back-end e API per la realizzazione del capitolato d'appalto, il team deve focalizzare il lavoro su un livello alto di implementazione per gestire la compatibilità grafica, derivante dall'uso della web app su diversi dispositivi.\\
Considerata la libertà di realizzazione offerta dai vari capitolati si è scelto di vagliare più approfonditamente tali proposte di capitolato d'appalto.

\subsection{Capitolato C4 - eBread: applicazione di lettura per dislessici}
\subsubsection{Descrizione}
Il capitolato si pone l'obiettivo di realizzare un'applicazione in ambiente \glossario{Android} che agevoli la lettura alle persone affette da dislessia, grazie all'aiuto di tecnologie appropriate, fra cui la sintesi vocale.\\
L'applicazione permetterebbe all'utente, tramite l'inserimento di un testo, varie modalità di interazione, per facilitare la lettura e diminuire il disagio dello stesso.

\subsubsection{Valutazione}
Nonostante l'utilità sociale del progetto sia innegabile, l'utilizzo di un servizio simile sarebbe comunque rivolto ad un mercato di nicchia e apparentemente non fornirebbe un vantaggio così rilevante, vista la mancanza di integrazione col resto del sistema.\\
Secondo il gruppo, infatti, questo porterebbe comunque ad un utilizzo forzato e complesso.

\subsection{Capitolato C6 - SWEDesigner: editor di diagrammi \glossario{UML} con generazione di codice}
\subsubsection{Descrizione}
Il capitolato prevede la realizzazione di un editor di diagrammi UML, con la relativa funzionalità di generazione di codice \glossario{Java} e \glossario{JavaScript}.\\
Lo sviluppo si basa soprattutto sul passaggio da diagramma UML a codice, studiandone nei dettagli tutte le caratteristiche e i casi che potrebbero verificarsi.

\subsubsection{Valutazione}
Al giorno d'oggi esistono già software simili, con la capacità di produrre del codice da diagrammi UML. La realizzazione di sistemi simili, cercando di ottimizzarne le funzionalità, è forse un obiettivo troppo esteso viste le risorse disponibili per questo progetto.\\
Il gruppo ha quindi trovato poco stimolante la creazione di un prodotto che rispecchia funzionalità già esistenti.
