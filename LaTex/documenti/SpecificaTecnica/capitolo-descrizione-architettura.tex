\section{Descrizione architettura}
Le tecnologie scelte, utilizzate per lo sviluppo del progetto \ProjectName{}, hanno fortemente influenzato le scelte progettuali e la conseguente architettura del sistema.\\

Il progetto si basa sull'utilizzo di Meteor.js, una piattaforma per lo sviluppo di applicazioni web attraverso l'uso del linguaggio JavaScript. JavaScript è un linguaggio volto alla programmazione orientata agli oggetti (\glossario{OOP}), ma che permette ai programmatori una notevole libertà sul metodo di implementazione dei pattern come l'\glossario{incapsulamento} e l'\glossario{ereditarietà}. A differenza di numerosi altri linguaggi rivolti alla programmazione ad oggetti, come C++, Java e suoi derivati, non esiste un costrutto esplicito con il quale il programmatore può definire classi. Tale limitazione è superata con lo standard 2015 di ECMAScript, il quale introduce il costrutto \textit{class} che automatizza il pattern OOP. La grande versatilità di JavaScript è dimostrata anche dalla considerevole quantità di progetti in cui viene scelto e utilizzato seguendo un approccio funzionale.\\

L'approccio scelto è stato principalmente Top-Down, in quanto le scelte principali sono state fatte basandosi sulla demo \DemoName{} e quindi sul prodotto finale da consegnare, analizzando e implementando nello specifico le funzionalità necessarie, adattandole al pattern MVC.\\

Data la duplice natura del progetto le due componenti sono state progettate separatamente, sfruttando per la Demo i risultati ottenuti dalla progettazione del framework. Per la componenti del framework non presentate nella demo si è usato invece un approccio di progettazione Bottom-Up, analizzando le funzionalità base a disposizione ed aggregandole per generare strumenti utili all'utente finale.