\section{Design pattern}
I \glossario{Design Pattern} descrivono la metodologia con cui affrontare problemi ricorrenti, fornendo degli standard che permettono di ottenere soluzioni eleganti e condivise.\\
La conoscenza dei Design Pattern favorisce la progettazione, il riuso e la manutenibilità del codice prodotto.\\
I principali Design Pattern vengono suddivisi in quattro categorie:
\begin{itemize}
	\item \textit{Architetturali}: affrontano il problema di progettazione di un sistema software fornendo uno schema di partenza su cui basare l'architettura;
	\begin{itemize}
		\item{MVC}
	\end{itemize}
	\item \textit{Creazionali}: affrontano il problema di astrarre il sistema rendendolo indipendente dall'implementazione concreta delle sue componenti;
	\item \textit{Strutturali}: affrontano il problema riguardante la composizione delle classi e degli oggetti, sfruttando l'ereditarietà e l'aggregazione;
	\item \textit{Comportamentali}: affrontano il problema dell'interazione tra i componenti, definendo la funzione degli oggetti e il modo in cui interagisce con gli altri.
\end{itemize}
Per una descrizione più approfondita e completa dei diversi Design Pattern utilizzati nella progettazione di \ProjectName{} si rimanda all'\appendice{app:design_pattern}. Di seguito viene invece descritto come sono stati utilizzati i diversi Design Pattern nella progettazione delle diverse parti dell'architettura.

\subsection{Design Pattern Architetturali}

\subsubsection{MVC - Model View Controller}
\textbf{Scopo:} permette di separare le responsabilità dei diversi componenti dell'applicazione dividendo presentazione, struttura e operazioni sui dati. Questo rende il codice manutenibile e di più semplice interpretazione.\\
\textbf{Utilizzo:} Viene utilizzato per lasciare la parte di gestione dell'interfaccia alla view lasciando al model e controller la gestione logica dell'applicazione, quindi storage dei dati, interazione tra gli utenti e aggiornamento bubble.

\subsection{Design Pattern Creazionali}

\subsubsection{Prototype}
\textbf{Scopo:} specifica il tipo di oggetti da creare utilizzando un'istanza \virgolette{prototipo} che viene copiata ed estesa.\\
\textbf{Utilizzo:} 

\subsubsection{Singleton}
\textbf{Scopo:} viene utilizzato per vincolare le classi che devono avere una sola istanza durante l'esecuzione dell'applicazione.\\
\textbf{Utilizzo:} viene utilizzato per l'interfaccia che gestisce la comunicazione col database MongoDB.

\subsection{Design Pattern Strutturali}

\subsubsection{Facade}
\textbf{Scopo:} fornisce un'interfaccia unica per un sottosistema più complesso, rendendo visibili solamente alcune parti agli altri oggetti, ed avendo un unico punto d'accesso minimizza le comunicazioni e le dipendenze.\\
\textbf{Utilizzo:} 


\subsubsection{Module design pattern}
\textbf{Scopo:} inserire negli oggetti encapsulation, rendendo \textit{private} e non accessibili dall'esterno campi e funzioni utilizzate come campi intermedi,.\\
\textbf{Utilizzo:} verrà utilizzato questo design qualora nella scrittura dell'oggetto debbano esserne rese private alcune parti.

\subsection{Design Pattern Comportamentali}

\subsubsection{Observer}
\textbf{Scopo:} permette l'aggiornamento di più viste contemporaneamente, disinteressandosi a quante o quali esse siano.\\
\textbf{Utilizzo:} verrà usato per sincronizzare le varie bubbles che opereranno con dati condivisi, in modo da mantenerle sempre aggiornate ed allineate.