\section{Introduzione}

\subsection{Scopo del documento}
Il presente documento ha lo scopo di definire la progettazione ad alto livello del progetto \glossario{\ProjectName{}}. Vengono quindi presentate le architetture del \glossario{framework} e delle bubble demo \glossario{To-do list} e Bubble \& eat con i relativi componenti software e i \glossario{Design Pattern} utilizzati per il loro sviluppo. Vengono inoltre descritte le diverse tecnologie che verranno utilizzate per lo sviluppo del progetto insieme al tracciamento tra le diverse componenti software e i requisiti a cui sono correlate.

\subsection{Scopo del Prodotto}
\ScopoDelProdotto

\subsection{Glossario}
\GlossarioIntroduzione

\subsection{Riferimenti}
\subsubsection{Normativi}
\begin{itemize}
	\item \textbf{\NormeDiProgetto}
	\item \textbf{\AnalisiDeiRequisiti}
	\item \textbf{Capitolato C5 - Monolith}: an interactive provider\\
	 \url{http://www.math.unipd.it/~tullio/IS-1/2016/Progetto/C5.pdf}
\end{itemize}
\subsubsection{Informativi}
\begin{itemize}
	\item \textbf{Seminario su Meteor e Rocket.Chat}\\ \url{http://www.math.unipd.it/~tullio/IS-1/2016/Progetto/C5.pdf}
	\item \textbf{\textit{Design Patterns, Elements of Reusable Object-Oriented Software} - Gamma, Helm, Johnson, Vlissides}:
	\begin{itemize}
		\item Creational Patterns;
		\item Structural Patterns;
		\item Behavioral Patterns.
	\end{itemize}
	\item \textbf{Slide dell'insegnamento Ingegneria del Software}:\\
	\url{http://www.math.unipd.it/~tullio/IS-1/2016/}
	\item \textbf{\textit{Software Engineering} - Ian Sommerville - 9th Edition (2011)}:
	\begin{itemize}
		\item Chapter 5: System modeling;
		\item Chapter 6: Architectural design;
		\item Chapter 7: Design and implementation;
		\item Chapter 16: Software reuse.
	\end{itemize}  
\end{itemize}
