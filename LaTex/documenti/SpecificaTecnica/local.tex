%%%%%%%%%%%%%%
%  COSTANTI  %
%%%%%%%%%%%%%%

% In questa prima parte vanno definite le 'costanti' utilizzate soltanto da questo documento.
% Devono iniziare con una lettera maiuscola per distinguersi dalle funzioni.

\newcommand{\DocTitle}{Specifica Tecnica}
\newcommand{\DocVersion}{\VersioneST{}}
\newcommand{\DocRedazione}{\nick \\ \tommy \\ \bea \\ \marco \\ \mattia \\ \lorenzo \\ \alice}
\newcommand{\DocVerifica}{\lorenzo \\ \mattia}
\newcommand{\DocApprovazione}{\mattia}
\newcommand{\DocUso}{Esterno}
\newcommand{\DocDistribuzione}{
	\Committente{} \\
	Gruppo \GroupName{} \\
	\Proponente{} 
}

% La descrizione del documento
\newcommand{\DocDescription}{
 Questo documento descrive l'architettura del framework \ProjectName{} e delle sue applicazioni dimostrative To-do list e \DemoName.
}

%%%%%%%%%%%%%%
%  FUNZIONI  %
%%%%%%%%%%%%%%
% Funzioni locali

\usepackage{xr}
\externaldocument[AdR-]{../AnalisiDeiRequisiti/capitolo-requisiti}

%comandi per i ruoli
\newcommand{\Manager}[1][1]{%
	\ifnum#1=1
	\textit{Manager}%
	\else
	\textit{Managers}%
	\fi
}
\newcommand{\Customer}[1][1]{%
	\ifnum#1=1
	\textit{Customer}%
	\else
	\textit{Customers}%
	\fi
}
\newcommand{\Deliveryman}[1][1]{%
	\ifnum#1=1
	\textit{Deliveryman}%
	\else
	\textit{Deliverymen}%
	\fi
}
\newcommand{\Purchasingmanager}[1][1]{%
	\ifnum#1=1
	\textit{Purchasing Manager}%
	\else
	\textit{Purchasing Managers}%
	\fi
}
\newcommand{\Chef}[1][1]{%
	\ifnum#1=1
	\textit{Chef}%
	\else
	\textit{Chefs}%
	\fi
}

\usepackage{subfig}