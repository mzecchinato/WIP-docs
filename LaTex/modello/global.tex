%%%%%%%%%%%%%%
%  COSTANTI  %
%%%%%%%%%%%%%%

% In questa prima parte vanno definite le 'costanti' utilizzate da due o più documenti.
% Serve a dare la giusta formattazione alle parole presenti nel glossario
% il nome del comando \glossary è già usato da LaTeX
\newcommand{\glossario}[1]{\textit{#1\ped{\ped{G}}}}

% Meglio non mettere gli \emph dentro le costanti, in certi casi creano problemi
\newcommand{\GroupName}{Or-bit}
\newcommand{\GroupEmail}{orbit.swe@gmail.com}
\newcommand{\ProjectName}{Monolith}
\newcommand{\ProjectVersion}{v1.0.0}

\newcommand{\DemoName}{Bubble \& eat}

\newcommand{\Proponente}{Red Babel}
\newcommand{\Committente}{Prof. Tullio Vardanega \\ Prof. Riccardo Cardin}
\newcommand{\CommittenteInline}{Prof. Tullio Vardanega, Prof. Riccardo Cardin}
\newcommand{\ResponsabileInCarica}{}

% La versione dei documenti deve essere definita qui in global, perchè serve anche agli altri documenti
\newcommand{\VersioneG}{1.0.0}
\newcommand{\VersionePQ}{1.0.0}
\newcommand{\VersioneNP}{1.0.0}
\newcommand{\VersionePP}{1.0.0}
\newcommand{\VersioneAR}{1.0.0}
\newcommand{\VersioneSF}{1.0.0}
\newcommand{\VersioneST}{0.0.1}
\newcommand{\VersioneMA}{0.0.0}
\newcommand{\VersioneMS}{0.0.0}
\newcommand{\VersioneMU}{0.0.0}
\newcommand{\VersioneDP}{0.0.0}
% Il verbale non ha versionamento.
% Lasciare vuoto, non mettere trattini o puntini
% Non sono permessi numeri nel nome di un comando :(
\newcommand{\VersioneVprimo}{} 
\newcommand{\VersioneVsecondo}{}

%ATTENZIONE: vengono rimossi dai comandi sottostanti i numeri di versione dei documenti non ancora esistenti
% Quando serve riferirsi a ``Nome del Documento + ultima versione x.y.z'' usiamo queste costanti:
\newcommand{\Glossario}{\emph{Glossario v\VersioneG{}}}
\newcommand{\PianoDiQualifica}{\emph{Piano di Qualifica v\VersionePQ{}}}
\newcommand{\NormeDiProgetto}{\emph{Norme di Progetto v\VersioneNP{}}}
\newcommand{\PianoDiProgetto}{\emph{Piano di Progetto v\VersionePP{}}}
\newcommand{\StudioDiFattibilita}{\emph{Studio di Fattibilità v\VersioneSF{}}}
\newcommand{\AnalisiDeiRequisiti}{\emph{Analisi dei Requisiti v\VersioneAR{}}}
\newcommand{\SpecificaTecnica}{\emph{Specifica Tecnica}}
\newcommand{\ManualeAdmin}{\emph{Manuale Admin}}
\newcommand{\ManualeSviluppatore}{\emph{Manuale Sviluppatore}}
\newcommand{\ManualeUtente}{\emph{Manuale Utente}}
\newcommand{\DefinizioneDiProdotto}{\emph{Definizione di Prodotto}}
\newcommand{\LetteraPresentazione}{\emph{Lettera di Presentazione}}

% Per riferirsi alle revisioni usare i comandi seguenti:
\newcommand{\RR}{Revisione dei Requisiti}
\newcommand{\RP}{Revisione di Progettazione}
\newcommand{\RQ}{Revisione di Qualifica}
\newcommand{\RA}{Revisione di Accettazione}

% Comandi utili per riferirsi alle varie fasi
\newcommand{\AR}{Analisi dei Requisiti}
\newcommand{\AD}{Analisi di Dettaglio}
\newcommand{\PA}{Progettazione Architetturale}
\newcommand{\PD}{Progettazione di Dettaglio}
\newcommand{\Cod}{Codifica}
\newcommand{\VV}{Verifica e Validazione}

% Comandi utili per riferirsi ai componenti del gruppo (Diari delle modifiche)
\newcommand{\alice}{Alice V. Sasso}
\newcommand{\bea}{Beatrice Guerra}
\newcommand{\lorenzo}{Lorenzo Ferrarin}
\newcommand{\marco}{Marco Ponchia}
\newcommand{\mattia}{Mattia Zecchinato}
\newcommand{\nick}{Nicola Dal Maso}
\newcommand{\tommy}{Tommaso Rosso}

% Comandi utili per riferirsi ai ruoli di progetto
\newcommand{\Responsabile}{\textit{Responsabile di Progetto}}
\newcommand{\Amministratore}{\textit{Amministratore}}
\newcommand{\Analista}{\textit{Analista}}
\newcommand{\Progettista}{\textit{Progettista}}
\newcommand{\Programmatore}{\textit{Programmatore}}
\newcommand{\Verificatore}{\textit{Verificatore}}

\newcommand{\Responsabili}{\textit{Responsabili di Progetto}}
\newcommand{\Amministratori}{\textit{Amministratori}}
\newcommand{\Analisti}{\textit{Analisti}}
\newcommand{\Progettisti}{\textit{Progettisti}}
\newcommand{\Programmatori}{\textit{Programmatori}}
\newcommand{\Verificatori}{\textit{Verificatori}}
\newcommand{\Redattori}{\textit{Redattori}}


\newcommand{\ScopoDelProdotto}{
Lo scopo di questo progetto è sviluppare un \glossario{framework} volto alla costruzione di \glossario{bubble} interattive per aggiungere funzionalità al programma di comunicazione \glossario{Rocket.Chat}.
Il progetto si propone anche di utilizzare il framework prodotto per fornire esempi significativi delle suddette bubble.
}

\newcommand{\GlossarioIntroduzione}{
Insieme al progetto il gruppo fornisce un glossario nel quale sono raccolti i termini tecnici, gli acronimi e le parole che necessitano di chiarimenti utilizzate nella documentazione.\\
I vocaboli presenti nel documento \Glossario{} sono marcati alla loro prima occorrenza con una \textit{\small{G}} al pedice.
}

\newcommand{\ToBeContinued}{\textit{Continua nella prossima pagina}}

% Altri comandi utili
\newcommand{\email}{email}
\newcommand*{\sezione}[1]{\hyperref[{#1}]{sezione §\ref*{#1}}}
\newcommand*{\appendice}[1]{\hyperref[{#1}]{Appendice \autoref*{#1}}}

%%%%%%%%%%%%%%
%  FUNZIONI  %
%%%%%%%%%%%%%%

% In questa seconda parte vanno definite le 'funzioni' utilizzate da due o più documenti.

% Comando per usare comodamente le virgolette
\newcommand{\virgolette}[1]{``#1''}

\newcommand*\flextt{%
  \fontdimen2\font=0.4em% interword space
  \fontdimen3\font=0.6em% interword stretch
  \fontdimen4\font=0.1em% interword shrink
  \fontdimen7\font=0.1em% extra space
  \hyphenchar\font=`\:
}

% Serve a dare la giusta formattazione al codice inline
\newcommand{\code}[1]{\flextt{\texttt{#1}}}

% Serve a dare la giusta formattazione a tutte le path presenti nei documenti
\newcommand{\file}[1]{\flextt{\texttt{#1}}}

% Permette di andare a capo all'interno di una cella in una tabella
\newcommand{\multiLineCell}[2][c]{\begin{tabular}[#1]{@{}l@{}}#2\end{tabular}}

% Per l'allineamento delle immagini
\usepackage[export]{adjustbox}

% Genera automaticamente la pagina di copertina
\newcommand{\makeFrontPage}{
  % Declare new goemetry for the title page only.
  \newgeometry{top=1cm}
  
  \begin{titlepage}
  \begin{center}

  \begin{center}
  \includegraphics[width=8cm]{../../modello/or-bit_bkg.png}
  \end{center}
  
%  \vspace{1cm}

  \begin{Huge}
  \textbf{\DocTitle{} \DocVersion{}}
  \end{Huge}
  
  \textbf{\emph{Gruppo \GroupName{} \, \texttwelveudash{} \, Progetto \ProjectName{}}}
  
  \vspace{10pt}

  \bgroup
  \def\arraystretch{1.3}
  \begin{tabular}{ r|l }
    \multicolumn{2}{c}{\textbf{Informazioni sul documento}} \\
    \hline
		% differenzia a seconda che \DocVersion{} stampi testo o no
		\setbox0=\hbox{\DocVersion{}\unskip}\ifdim\wd0=0pt
			% nulla (non ho trovato come togliere l'a capo)
			\\
		\else
			\textbf{Versione} & \DocVersion{} \\
		\fi
    \textbf{Redazione} & \multiLineCell[t]{\DocRedazione{}} \\
    \textbf{Verifica} & \multiLineCell[t]{\DocVerifica{}} \\
    \textbf{Approvazione} & \multiLineCell[t]{\DocApprovazione{}} \\
    \textbf{Uso} & \DocUso{} \\
    \textbf{Distribuzione} & \multiLineCell[t]{\DocDistribuzione{}} \\
  \end{tabular}
  \egroup

  \vspace{10pt}

  \textbf{Descrizione} \\
  \DocDescription{}  

%  \vspace{0.2cm}
  

  \end{center}
  \begin{figure}[H]
	\includegraphics[height=3.7cm, right]{../../modello/monolith}
  \end{figure}
  \end{titlepage}
  
  % Ends the declared geometry for the titlepage
  \restoregeometry
}
